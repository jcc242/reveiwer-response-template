%%%%%%%%%%%%%%%%%%%%%%%%%%%%%%%%%%
%%                              %%
%% Introduction                 %%
%%                              %%
%%%%%%%%%%%%%%%%%%%%%%%%%%%%%%%%%%
\documentclass[preprint,12pt]{elsarticle}

%% Use the option review to obtain double line spacing
%% \documentclass[preprint,review,12pt]{elsarticle}

%% Use the options 1p,twocolumn; 3p; 3p,twocolumn; 5p; or 5p,twocolumn
%% for a journal layout:
%% \documentclass[final,1p,times]{elsarticle}
%% \documentclass[final,1p,times,twocolumn]{elsarticle}
%% \documentclass[final,3p,times]{elsarticle}
%% \documentclass[final,3p,times,twocolumn]{elsarticle}
%% \documentclass[final,5p,times]{elsarticle}
%% \documentclass[final,5p,times,twocolumn]{elsarticle}

%% if you use PostScript figures in your article
%% use the graphics package for simple commands
%% \usepackage{graphics}
%% or use the graphicx package for more complicated commands
\usepackage{graphicx}
%% or use the epsfig package if you prefer to use the old commands
\usepackage{color}
\usepackage{xspace}
\usepackage{algorithmic}
\usepackage{amsmath}
%% The amssymb package provides various useful mathematical symbols
\usepackage{amssymb}
%% The amsthm package provides extended theorem environments
%% \usepackage{amsthm}
\usepackage{wrapfig}
\usepackage{pdfpages}
\usepackage{siunitx}

% and a few extras
\newcommand{\bbold}{{\boldsymbol{b}}}
\newcommand{\Ubold}{{\boldsymbol{U}}}
\newcommand{\Wbold}{{\boldsymbol{W}}}
\newcommand{\Nbold}{{\boldsymbol{N}}}
\newcommand{\Xbold}{{\boldsymbol{X}}}
\newcommand{\xibold}{{\boldsymbol{\xi}}}
\newcommand{\vecnabla}{{\vec{\nabla}}}
\newcommand{\nablabold}{{\boldsymbol{\nabla}}}
\newcommand{\Deltabold}{{\boldsymbol{\Delta}}}
\newcommand{\eboldd}{{\ebold^d}}
\newcommand{\ebolds}{{\ebold^s}}

%\renewcommand{\ivec}{{\vec{\imath}}}
%\renewcommand{\jvec}{{\vec{\jmath}}}
%\renewcommand{\uvec}{{\vec{u}}}
%\renewcommand{\xvec}{{\vec{x}}}
%\renewcommand{\evec}{{\vec{e}}}
%\newcommand{\xivec}{{\vec{\xi}}}
%\newcommand{\evecd}{{\vec{e}^{\,d}}}
%\newcommand{\Nvvec}{{\Vec{\Vec{N}}}}
%\newcommand{\NvvecT}{{\Vec{\Vec{N}}\vphantom{N}^T}}
%\newcommand{\NvvecTsub}[1]{{\Vec{\Vec{N}}\vphantom{N}^T_{#1}}}
%\newcommand{\nablavec}{{\vec{\nabla}}}
\newcommand{\NboldT}{{\Nbold^T}}
\newcommand{\NrmT}{{\rm{N}}^{T}}
\newcommand{\Nrm}{{\rm{N}}}
\newcommand{\Frm}{{\text{F}}}
\newcommand{\Fbrm}{{\text{\bfseries F}}}
\newcommand{\Ubrm}{{\text{\bfseries U}}}
\newcommand{\Wbrm}{{\text{\bfseries W}}}
\newcommand{\Sbrm}{{\text{\bfseries S}}}
\newcommand{\Nc}{{\mathcal{N}}}
\newcommand{\xb}{{\boldsymbol{x}}}
\newcommand{\Xb}{{\boldsymbol{X}}}
\newcommand{\xib}{{\boldsymbol{\xi}}}
\newcommand{\vecx}{{\vec{x}}}
%\newcommand{\Xvec}{{\vec{X}}}
\newcommand{\xivec}{{\vec{\xi}}}
\newcommand{\Nb}{{\boldsymbol{N}}}
\newcommand{\Fb}{{\boldsymbol{F}}}

\newcommand{\zb}{{\boldsymbol{z}}}
\newcommand{\Zb}{{\boldsymbol{Z}}}
\newcommand{\zetab}{{\boldsymbol{\zeta}}}
\newcommand{\Mb}{{\boldsymbol{M}}}
\newcommand{\Mc}{{\mathcal{M}}}
\newcommand{\Gb}{{\boldsymbol{G}}}

\newcommand{\ib}{{\boldsymbol{i}}}
\newcommand{\eb}{{\boldsymbol{e}}}
\newcommand{\dpr}{{d^\prime}}
\newcommand{\Gbrm}{{\text{\bfseries G}}}

\newcommand{\Subitem}[1]{
  \begin{itemize}
  \item{#1}
  \end{itemize}
  }
%%%%%%%%%%%%%%%%%%%%%%%%%%%%%%%%%%
%%                              %%
%% New commands                 %%
%%                              %%
%%%%%%%%%%%%%%%%%%%%%%%%%%%%%%%%%%

\makeatletter
 \begin{document}
% 

\noindent
Dear Editor-in-Chief:

\vspace{6pt}
\noindent
We'd like to thank our reviewers for the comments, suggestions, and advice on
improving our paper, titled ``My Awesome Paper's Name.''
We will address the comments point by point in the sections below.
The feedback provided has improved the quality of the paper.
Please let us know if any questions remain and we will be pleased to answer
them.
We appreciate the time and effort dedicated to reviewing our paper. 

\vspace{4ex}
\hspace{-1ex}{\large{\textbf {I. Response to Reviewer 1}}}
\begin{enumerate}

  \item \textit{This is reviewer one's first comment.}
        \vspace{2ex}

        This is the response to reviewer one's first comment.
        Here is a bulleted list of some changes:
        \begin{itemize}
          \item We changed the wording to make things more clear, pages 3--4, lines 181--253.
          \item Addition of a new section titled ``My New Section'', beginning on page 4, line 254.
                This section provides a detailed description of something the reviewer wanted.
        \end{itemize}

        \vspace{2ex}
  \item \textit{This is reviewer one's second comment.}
        \vspace{2ex}

        Don't forget to include a vspace of size 2ex between each comment and response.
        Also, here is how I cite a paper for the reviewer:
        \begin{itemize}
          \item Falgout, R.D., Katz, A., Kolev, Tz.V., Schroder, J.B., Wissink,
                A.M., Yang, U. M.: Parallel Time Integration with Multigrid Reduction
                for a Compressible Fluid Dynamics Application, Technical Report, 2014,
                LLNL-JRNL-663416. https://github.com/XBraid/xbraid/wiki/Project-Publications
        \end{itemize}

        \vspace{2ex}
\end{enumerate}


\newpage
\hspace{-1ex}{\large{\textbf {II. Response to Reviewer 2}}}
\begin{enumerate}

  \item \textit{This is the second reviewer's first comment}
        \vspace{2ex}

        Here is a response the reviewer's comment.

        \vspace{2ex}
  \item \textit{Here is a second comment by the second reviewer.}
        \vspace{2ex}

        Don't forget to cite page and line numbers for changes you make.
        For example, we addressed the reviewer's concerns on page 5, lines 243--360.

        \vspace{2ex}
\end{enumerate}


\end{document}
